Magnetic systems often exhibit nonlinear responses to external magnetic fields, which means that the correlation between an applied magnetic field and the resulting magnetization is not proportional, which can induce a variety of interesting phenomena, such as chaotic behavior, autoressonance, and turbulance in earth\textquotesingle s magnetotail~\cite{nonlinearResponse,Bratman1983,Loeb1986,VerandaM,Brunton2017,Chang1999}. 

%Nonlinearity can arise due to various factors such as magnetic anisotropy, domain wall motion, and magnetic interactions~\cite{HORVATH2022119279}.

This work focuses on investigating the dynamics of a system composed of three equidistantly spaced bar magnets affixed to a table, each free to rotate about its center. Starting from an initial equilibrium state, the system exhibits unexpected and chaotic behavior when a small increase in kinetic energy is applied to a single magnet, resulting in a small initial spin. This behavior depends on specific values of the physical parameters, such as the dipole moments and moments of inertia of the dipoles. The mechanism by which kinetic energy transfers among the magnets in the system is not fully understood. Understanding this phenomenon is crucial for gaining insights into the dynamics of magnet systems and their potential applications.

% This part may by more usefull in the end of introduction:
In order to better analyze this system, we consider the length of the magnets to be small compared to the distance between them, allowing us to approximate the bar magnets for magnetic dipoles.

%This is very important for the case we investigate only a dipole and a external magnetic field
A system composed of two magnetic dipoles exhibits a nonlinear coupling, and presents two differents time scales depending on the magnitudes of the magnetic interactions, because they are of different nature~\cite{LAROZE20081440}. 

Previous works shown that the influence of small fluctuations in a system composed of two magnetic dipoles can lead to two different behaviors: the dipoles fluctuate around stable fixed points (with low amplitude fluctuations), or stochastic reversals occours between stable fixed points (with strong fluctuations). Low energy fluctuations lead to disjoint basins of attraction near stable fixed points, while higher fluctuations connect basins (including an unstable fixed point) with stochastic reversals and Poisson-distributed waiting time~\cite{StochasticReversalDynamics}. 

\subsection{The forced dipole}
The magnetic field generated by a single dipole is given by the formula: 
\begin{equation}
    \boldsymbol{B} = \dfrac{3\mu_0}{4 \pi r^3}\bigg[ (\boldsymbol{m \cdot \hat{r}})\boldsymbol{\hat{r}} - \dfrac{1}{3}\boldsymbol{m} \bigg],
    \label{eq:MagneticFieldDipole}
\end{equation}
where $\boldsymbol{m}$ is the magnetic moment of the dipole and $\boldsymbol{r}$ is the distance to the dipole. 
\begin{equation}
    \boldsymbol{m} = m (\cos (\theta),\sin (\theta),0),
    \label{eq:MagneticMoment}
\end{equation}
When a fixed dipole, with magnetic moment given by the equation~\ref{eq:MagneticMoment}, is free to spin around it\textquotesingle s center, and it\textquotesingle s is placed on a surfice and subjected to a homogeneous magnetic field parallel to the surfice, represented by the equation:
\begin{equation}
    \boldsymbol{B} = B (\cos (\theta_1),\sin (\theta_1),0),
\end{equation}
the dipoles experiences a torque, given by the Newton\textquotesingle s second law of motion for spinning objects: 
\begin{equation}
    \boldsymbol{\tau} = \boldsymbol{m} \times \boldsymbol{B}_{ext}
    \label{eq:NewtonSecondLaw}
\end{equation}
Note that $\boldsymbol{B}_{ext}$ and $\boldsymbol{m}$ are vectors in the same plane, and therefore the resulting torque $\tau$ is perpendicular to both of them and has components only in the $\boldsymbol{\hat{z}}$ direction: 
\begin{equation}
        \tau = mB_{ext} \sin(\theta - \theta_1)
        \label{eq:SingleDipoleEquation},
\end{equation}
The equation of motion for the dipole is then:
\begin{equation}
    \begin{aligned}
        I\ddot{\theta} - m B \sin(\theta - \theta_1) &=0\\
        \ddot{\theta} -\omega^2\sin(\theta - \theta_1) &=0
    \end{aligned}
    \label{eq:DipoleDiffEquation}
\end{equation}
where $I$ is the moment of inertia of the dipole, and $\omega$ its natural frequency.
If we make the direction of the external magnetic field vary in time, in such a way that:
\begin{equation}
    \theta_1 = \varepsilon \sin(\Omega t),
\end{equation}
then equation~\ref{eq:DipoleDiffEquation} becomes: 
\begin{equation}
    \ddot{\theta} -\omega^2\sin(\theta - \varepsilon \sin(\Omega t)) =0,
\end{equation}
Now, the equation~\ref{eq:DipoleDiffEquation} still resembles the pendulum equation~\ref{eq:Pendulum}, but the dynamics of the system has change significantly, specially for some values of $\Omega$, and $\omega$. 
\begin{equation}
    \ddot{\theta} - \omega^2 \sin(\theta)
    \label{eq:Pendulum}
\end{equation}
Our work will try to understand how those values affect the dynamics of this simple system. 
% was extensivelly studied in~\cite{yung1970analytic, ku2016interaction, santos2019dinamica}, in~\cite{yung1970analytic} equations for the motion of two dipoles are derived, and~\cite{ku2016interaction, santos2019dinamica} investigated the damping version as well as the system in a presence of a external magnetic field.



% As any dynamical system, there could be stable or unstable points of equilibrium for this system, and as shown in the work of Carlos for the system with two magnetic dipoles there are eigth equilibrium points, and for a ring made out of magnetic dipoles.