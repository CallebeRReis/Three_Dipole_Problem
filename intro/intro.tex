This work focuses on investigating the dynamics of a system composed of three equidistantly spaced bar magnets affixed to a table, each free to rotate about its center. Starting from an initial equilibrium state, the system exhibits unexpected and chaotic behavior when a small increase in kinetic energy is applied to a single magnet, resulting in a small initial spin. This behavior depends on specific values of the physical parameters, such as the dipole moments and moments of inertia of the dipoles. The mechanism by which kinetic energy transfers among the magnets in the system is not fully understood. Understanding this phenomenon is crucial for gaining insights into the dynamics of magnet systems and their potential applications. 

The system composed of two magnetic dipoles was extensivelly studied in~\cite{yung1970analytic, ku2016interaction, santos2019dinamica}, in~\cite{yung1970analytic} equations for the motion of two dipoles are derived, and~\cite{ku2016interaction, santos2019dinamica} investigated the damping version as well as the system in a presence of a external magnetic field. 

To analyze this system, we consider the length of the magnets to be small compared to the distance between them, allowing us to approximate it using simple magnetic dipoles. By employing the equations for the torque acting on a magnetic dipole in an external magnetic field, we can gain insights into the dynamics of this system.

As any dynamical system, there could be stable or unstable points of equilibrium for this system, and as shown in the work of Carlos for the system with two magnetic dipoles there are eigth equilibrium points, and for a ring made out of magnetic dipoles 
