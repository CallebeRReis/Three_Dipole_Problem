Magnetic systems often exhibit nonlinear responses to external magnetic fields, which means that, the correlation between an applied magnetic field and the resulting magnetization is not proportional, which can induce a variety of interesting phenomena, such as chaotic behavior, autoressonance, and turbulance in earth\textquotesingle s magnetotail~\cite{nonlinearResponse,Bratman1983,Loeb1986,VerandaM,Brunton2017,Chang1999}. 

%Nonlinearity can arise due to various factors such as magnetic anisotropy, domain wall motion, and magnetic interactions~\cite{HORVATH2022119279}.

This work focuses on investigating the dynamics of a system composed of three equidistantly spaced bar magnets affixed to a table, each free to rotate about its center. In order to better analyze this system, we consider the length of the magnets to be small compared to the distance between them, allowing us to approximate the bar magnets for magnetic dipoles.

We studied the case where two magnets are much stronger than the third one, and, taking sufficient approximations, we were able to investigate the system as a single magnet in a homogeneous magnetic field that oscillates in direction.

From an initial equilibrium state, the simplified system can exhibit unexpected and chaotic behavior when a small increase in kinetic energy is applied to the small magnet through the oscillations of the magnetic field. This interesting behavior occurs at specific values of the physical parameters, such as the dipole moment and the frequency at which the field oscillates.

Understand the mechanism by which kinetic energy transfers among the magnetic field and the magnet was the motivation to this project. Having a good understanding of this phenomenon is crucial for gaining insights into the dynamics of magnetic systems and their potential applications.

%This is very important for the case we investigate only a dipole and a external magnetic field
One may suggest that, for a system composed of two magnetic dipoles there exists a nonlinear coupling, and the problem presents two different time scales depending on the magnitudes of the magnetic interactions, because they are of different nature~\cite{LAROZE20081440}. As we\textquotesingle ll see, such a phenomenon occurs for the single magnetic dipole in an oscillating magnetic field. 

Previous works shown that the influence of small fluctuations in a system composed of two magnetic dipoles can lead to two different behaviors: the dipoles fluctuate around stable fixed points (with low amplitude fluctuations), or stochastic reversals occours between stable fixed points (with strong fluctuations). Low energy fluctuations lead to disjoint basins of attraction near stable fixed points, while higher fluctuations connect basins (including an unstable fixed point) with stochastic reversals and Poisson-distributed waiting time~\cite{StochasticReversalDynamics}. In this work, we are interested in the second case, by each the system presents a rather unexpected behavior.  

% was extensivelly studied in~\cite{yung1970analytic, ku2016interaction, santos2019dinamica}, in~\cite{yung1970analytic} equations for the motion of two dipoles are derived, and~\cite{ku2016interaction, santos2019dinamica} investigated the damping version as well as the system in a presence of a external magnetic field.



% As any dynamical system, there could be stable or unstable points of equilibrium for this system, and as shown in the work of Carlos for the system with two magnetic dipoles there are eigth equilibrium points, and for a ring made out of magnetic dipoles.