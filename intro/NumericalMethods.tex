\section{Numerical Methods}
\subsection{Runge-Kutta 4th order}~\label{sec:RK4}

In order to investigate equation (\ref{eq:EquationOfMotion}) numerically, we used the iterative method of Runge-Kutta of fourth order (RK4), which has a low computational cost, and fourth order accuracy, i.e, at each iteration the error is at most $\mathcal{O}(h^4)$, where $h$ is the integration step.

Before applying the method, we have transformed the second order differential equation of motion of dipole (\ref{eq:EquationOfMotion}), into two first order differential equation by making the transformations:

\begin{equation}
    \begin{aligned}
        x_1 &= \theta,\\
        x_2 &= \dot{\theta}, 
    \end{aligned}
\end{equation}
leading to the system of first order differential equations below:
\begin{equation}
    \begin{cases}
        \dot{x_1} &= x_2,\\
        \dot{x_2} &= -\eta^2 \sin(x_1 - \varepsilon \sin(\Omega t)),
    \end{cases}
\end{equation}
We used $h=10^{-4}$ for most simulations, equation (\ref{eq:EquationOfMotion}) and $h = 0.05$ for the construction of the bifurcation diagrams.

\subsection{Simpson\textquotesingle s 1/3 rule and Fourier transform}
After solving the differential equation numerically using the \hyperref[sec:RK4]{RK4} method, we used the Simpson\textquotesingle s $1/3$ rule, equation (\ref{eq:SimpsonsRule}), to numerically integrate the Fourier transform, equation~\ref{eq:FourierTransform}.
\begin{equation}
    \mathcal{F}\{f(t)\}= \dfrac{2}{T}\int_{0}^{T} f(t) e^{- i \xi t}dt
    \label{eq:FourierTransform}
\end{equation}
% \begin{equation}
    \begin{multline}
        \int_{a}^{b} f(x) dx = \dfrac{1}{3} h\bigg[ f(a) + 
         4\sum^{n/2}_{n=1}f(x_{2i-1}) \\
         + 2 \sum^{n/2 - 1}_{n=1}f(x_{2i})+f(b) \bigg]        
    \label{eq:SimpsonsRule}
\end{multline}
\subsection{Bifurcation diagrams}

For creating the bifurcation diagrams, figures~\ref{fig:bif005} and~\ref{fig:bif05}, we vary the frequency of the magnetic field, $\Omega$, from $-3$ to $3$ using a step of $0.01$, and for each frequency $\Omega$ we solved equation (\ref{eq:EquationOfMotion}) for $t \in [0,500]$ using the \hyperref[sec:RK4]{RK4} method, then applied the Fourier transform and filtered out the frequencies with amplitude less than $20\%$ of the maximum amplitude. Finally, for each $\Omega$ we plotted the frequencies obtained from the Fourier transform, $\omega$, creating a graph, $(\Omega,\omega)$ where the $\omega$ frequencies represent the frequency response of the system.  

