\section{Numerical Methods}

\subsection{Runge-Kutta 4th order}
\label{sec:RK4}
In order to investigate the equation (\ref{eq:EquationOfMotion}) numerically, we used the iterative method of Runge-Kutta of fourth order (RK4), which presents a low computational cost, and fourth order accuracy, i.e, at each iteration the error is $\mathcal{O}(h^4)$, where $h$ is the integration step.

Before applying the method, we have transformed the second order differential equation of motion of dipole (\ref{eq:EquationOfMotion}), into two first order differential equation by making the transformations:

\begin{equation}
    \begin{aligned}
        x_1 &= \theta,\\
        x_2 &= \dot{\theta}, 
    \end{aligned}
\end{equation}
leading to the system of first order differential equations below:
\begin{equation}
    \begin{cases}
        \dot{x_1} &= x_2,\\
        \dot{x_2} &= -\omega^2 \sin(x_1 - \varepsilon \sin(\Omega t)),
    \end{cases}
\end{equation}
\subsection{Simpson\textquotesingle s 1/3 rule and Fourier transform}
After solving the differential equation numerically using the \hyperref[sec:RK4]{RK4} method, we used the Simpson\textquotesingle s $1/3$ rule, to numerically integrate the Fourier transform: 
\begin{equation}
    \dfrac{2}{T}\int_{0}^{T} f(t) e^{- i \xi t}dt
\end{equation}
