\section{First Results}

In this Section, we\textquotesingle ll briefly discuss the dynamics of the equation (\ref{eq:EquationOfMotion}), with most numerical results. 

The dipole\textquotesingle s dynamics was studied for small amplitude of oscillation of the magnetic field, we used two main values for the parameter $\varepsilon$, $\varepsilon = 0.05$ and $\varepsilon = 0.5$, and the following initial conditions for both cases:
\begin{equation}
    \begin{cases}
        \theta (0) &= 1\\        
        \dot{\theta}(0) &= 0
    \end{cases}
\end{equation} 
In the former case, $\varepsilon = 0.05$, we verify that the dipole\textquotesingle s movement was periodic for all the values of the frequency of the magnetic field $\Omega$ between $-3.0$ and $3.0$, and as expected, the dipole behaved like a simple pendulum. 

However, for the latter case, $\varepsilon = 0.5$, the dipole\textquotesingle s movement was much more complex, and for some subintervals in the frequency range in $[-3,3]$, the movement was not periodic, and maybe not even limited. 

\begin{figure}[h]
    \scalebox{0.7}{\input{FrequencyBifurcation005.tex}}
    \caption{Bifurcation diagram of frequencies from the oscillating magnetic field $\Omega$ and the response frequencies on the dipole $\omega$, for the case with $\varepsilon = 0.05$.}
    \label{fig:big005}
\end{figure}

\begin{figure}[h]
    \scalebox{0.7}{\input{FrequencyBifurcation.tex}}
    \caption{Bifurcation diagram of frequencies from the oscillating magnetic field $\Omega$ and the response frequencies on the dipole $\omega$, for the case with $\varepsilon = 0.5$.}
    \label{fig:bif05}
\end{figure}
