
\subsection{Theoretical model}

A single magnetic dipole with magnetic moment $\boldsymbol{m}$, generates a magnetic field at a distance $\boldsymbol{r}$, accordingly to the expression: 
\begin{equation}
    \boldsymbol{B} = \dfrac{3\mu_0}{4 \pi r^3}\bigg[ (\boldsymbol{m \cdot \hat{r}})\boldsymbol{\hat{r}} - \dfrac{1}{3}\boldsymbol{m} \bigg],
    \label{eq:MagneticFieldDipole}
\end{equation}
where, $\mu_0$ is the magnetic permeability of the medium~\cite{jackson1999classical}. 

If subjected to an external magnetic field $\boldsymbol{B}_{ext}$, the magnetic dipole experiences a torque given by: 
\begin{equation}
    \boldsymbol{\tau} = \boldsymbol{m} \times \boldsymbol{B}_{ext}.
    \label{eq:NewtonSecondLaw}
\end{equation}

Let us consider a system composed of three magnetic dipoles, each placed at a vertex of an equilateral triangle and equally spaced. The magnetic field generated by each of the dipoles is felt by the other two. By adding the torques at the dipole $i$, generated by the dipoles $j$ and $k$, we find that, the resulting torque on dipole $i$ is:
\begin{equation}
    \begin{aligned}
        \boldsymbol{\tau}_i &= \boldsymbol{\tau}_{ij} + \boldsymbol{\tau}_{ik},\\
            & =  \boldsymbol{m}_i \times \boldsymbol{B}_{j i}+\boldsymbol{m_i} \times \boldsymbol{B}_{k i},\\
            &= \boldsymbol{m}_i \times \boldsymbol{B}_{res},
    \end{aligned}
    \label{eq:Torques}
\end{equation}
where, $\boldsymbol{B}_{res}$, represents the resulting magnetic field generated by the two dipoles $j$ and $k$. 

In this work, we analyze a simplified model where two dipoles have much greater moments of inertia than the third, and that the resulting magnetic field generated by the two heavy dipoles is approximately homogeneous, i.e,

\begin{equation}
    \begin{aligned}
        % I_1 & \ll I_2,\\
        % I_1 & \ll I_3,\\
        \boldsymbol{B}_{res} &= B(\cos(\theta_1),\sin(\theta_1), 0),
    \end{aligned}
    \label{eq:Hipotesys}
\end{equation}
where $\theta_1$ represents the angular displacement of the magnetic field.

If the smaller dipole has magnetic moment given by:
\begin{equation}
    \boldsymbol{m}_1 = m_1 (\cos(\theta), \sin(\theta), 0),
    \label{eq:MagneticMoment}
\end{equation}
then, because $\boldsymbol{B}_{res}$ and $\boldsymbol{m_1}$ are in the same plane, the torque at the smaller dipole will have only components in the direction of $\boldsymbol{\hat{z}}$.

Therefore, in the above conditions, and accordingly to Newton~\textquotesingle s Second Law of motion for spinning objects, the equation~(\ref{eq:Torques}) gives rise to the equation of motion of a single dipole:
\begin{equation}
    % \begin{aligned}
        \tau_1 - I_1 \ddot{\theta} = 0,
\end{equation}
from where, 
\begin{equation}    
        I \ddot{\theta} + m_1 B \sin(\theta - \theta_1) = 0.        
    % \end{aligned}
\end{equation}s 
Now calling
\begin{equation}
    \eta^2 = m_1 B/I,
\end{equation} 
then
\begin{equation}
    \ddot{\theta} +\eta^2 \sin(\theta - \theta_1) = 0.
    \label{eq:EquationOfMotion}
\end{equation}
We remark that equation (\ref{eq:EquationOfMotion}) is very similar to the pendulum equation, except for the $\theta_1$ parameter.

% \subsection{Forced System}

In general, the two large dipoles do not remain static, but oscillate with small amplitude:
\begin{equation}
    \theta_1(t) = \varepsilon \sin(\Omega t).
\end{equation}
Therefore, generating an oscillating magnetic field, with frequency $\Omega$, which turns equation (\ref{eq:EquationOfMotion}) into the equation (\ref{eq:MotionWithOmega}).
\begin{equation}
    \ddot{\theta} +\eta^2 \sin(\theta - \varepsilon \sin(\Omega t)) = 0.
    \label{eq:MotionWithOmega}
\end{equation}
This is the final equation that will be studied in this article.
% Making the following change of variables:
% \begin{equation}
%     \begin{aligned}
%         \phi &= \theta - \varepsilon \sin(\Omega t),\\
%         \ddot{\phi} &= \ddot{\theta} + \varepsilon \Omega^2 \sin(\Omega t)
%     \end{aligned}
% \end{equation}
% resulting in, 
% \begin{equation}
%     \ddot{\phi} + \eta^2 \sin(\phi) =\Omega^2 \varepsilon \sin(\Omega t)
% \end{equation}
% which is the equation of a forced pendulum.

% (to be continued)
