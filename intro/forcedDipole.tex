
\subsection{The triple dipole}
A single magnetic dipole with magnetic moment $\boldsymbol{m}$, generates a magnetic field, at a distance $\boldsymbol{r}$, accordindly to the expression: 
\begin{equation}
    \boldsymbol{B} = \dfrac{3\mu_0}{4 \pi r^3}\bigg[ (\boldsymbol{m \cdot \hat{r}})\boldsymbol{\hat{r}} - \dfrac{1}{3}\boldsymbol{m} \bigg],
    \label{eq:MagneticFieldDipole}
\end{equation}
where, $\mu_0$ is the magnetic permeability of the medium. 

When subjected to an external magnetic field $\boldsymbol{B}_{ext}$, it experiences a torque following the expression: 
\begin{equation}
    \boldsymbol{\tau} = \boldsymbol{m} \times \boldsymbol{B}_{ext}.
    \label{eq:NewtonSecondLaw}
\end{equation}

Now consider a system composed of three magnetic dipoles, each placed at a vertex of an equilateral triangle and therefore equally spaced. The magnetic field generated by each of the dipoles is felt by the other two, and at each dipole, and adding the torques at the dipole $i$, generated by the dipoles $j$ and $k$, we find that the resulting torque is:
\begin{equation}
    \begin{aligned}
        \boldsymbol{\tau}_i &= \boldsymbol{\tau}_{ij} + \boldsymbol{\tau}_{ik}\\
            & =  \boldsymbol{m}_i \times \boldsymbol{B}_{j i}+\boldsymbol{m_i} \times \boldsymbol{B}_{k i}\\
            &= \boldsymbol{m}_i \times \boldsymbol{B}_{res}
    \end{aligned}
    \label{eq:Torques}
\end{equation}
where, $\boldsymbol{B}_{res}$, represents the resulting magnetic field generated by the other two dipoles $j$ and $k$. 

Considering the special case where there are two strong dipoles, and one weak one, with magnetic moment given by:
\begin{equation}
    \boldsymbol{m_1} = m_1 (\cos(\theta), \sin(\theta), 0),
\end{equation}
and that the resulting magnetic field generated by the two strong dipoles is approximately homogeneous, i.e, under the follwing conditions:
\begin{equation}
    \begin{aligned}
        m_1 & \ll m_2\\
        m_1 & \ll m_3\\
        \boldsymbol{B}_{res} &= B(\cos(\theta_1),\sin(\theta_1), 0)
    \end{aligned}
    \label{eq:Hipotesys}
\end{equation}
Note that $\boldsymbol{B}_{res}$ and $\boldsymbol{m_1}$ are in the same plane, and therefore the resulting torque $\boldsymbol{\tau}$ has components only in the $\boldsymbol{\hat{z}}$ direction. 

Therefore, in the above conditions, and acoordindly to Newton~\textquotesingle s Second Law of motion for spinning objects, the equation~(\ref{eq:Torques}) gives rise to the equation of motion of a single dipole:
\begin{equation}
    \begin{aligned}
        &\tau_1 - I_1 \ddot{\theta} = 0\\
        &I \ddot{\theta} + m_1 B \sin(\theta - \theta_1) = 0        
    \end{aligned}
\end{equation}
Calling,
\begin{equation}
    \omega^2 = m_1 B/I 
\end{equation} 
then
\begin{equation}
    \ddot{\theta} +\omega^2 \sin(\theta - \theta_1) = 0
\end{equation}

